\documentclass[a4paper]{article}

\title{Ising Model: Single spin flip dynamics}
\author{Mehmet S\"uzen, Ph.D. \footnote{mehmet.suzen@physics.org}}

\usepackage{Sweave}
\begin{document}
\maketitle

The Ising-Lenz Model \footnote{E. Ising 1924} appear as one of the land mark systems in statistical physics,
as well as in many other fields including computational neuroscience. Because of its simplicity and success
in explaining critical phenomenon, Ising Model is routinely used in research and as a teaching concept in
statistical mechanics and in standard Monte Carlo methods. There exist analytical solutions for 1D and 2D
Ising Model. Still, numerical solutions with MC could provide insights.

{\bf isingLenzMC} package provides utilities to simulate one dimensional Ising Model with Metropolis Monte 
Carlo with single flip dynamics in periodic boundary conditions \footnote{Higher dimensional models may be 
introduced in future releases.}. Computationally intensive parts are written in {\it low-level language (C)} 
for efficiency reasons.

\section{Ising Model}

Consider one dimensional lattice that contains $N$ sites. Each site values can be labelled as $\{s_{i}\}_{i=1}^{N}$. 
In two state version of a lattice, which is an Ising Model, sites can take two values, such as $\{1,-1\}$,
corresponding to spin up and spin down states, for example as a model of magnetic material or the state of a 
neuron.

The total energy, so called Hamiltonian, of the system can be expressed as follows, for short-range parts with 
interaction strength $J$:

$$ H(\{s_{i}\}_{i=1}^{N}, J, H) = J \big( (\sum_{i=1}^{N-1} s_{i} s_{i+1}) + (s_{1} s_{N}) \big) + H \sum_{1}^{N} s_{i} $$ 

This expression contains two interactions, due to nearest-neighbors (NN) and due to an external field. Coefficients 
$J$ and $H$ corresponds to these interactions respectively. Note that, additional term in NN interactions $s_{1} s_{N}$
appears due to periodic (cyclic) boundary conditions.


\section{Simulation of the Ising Model}

\subsection{Single-flip dynamics}
One of the common ways to generate dynamics for a lattice system explained in the previous section is changing
the value of a randomly choosen site to its opposite value as a dynamical step. This procedure is called single-flip
dynamics. An example dynamics for 5 site lattice can be as follows:

$$\{1, -1, 1, 1, -1\}$$
$$\{1, -1, 1, -1, -1\}$$
$$\{1, 1, 1, -1, -1\}$$
$$\{-1, 1, 1, -1, -1\}$$

Note that, the quality of this kind of dynamics depends on the quality of the random number generator (RNG) we use in selecting
flipped site. However, we assume that RNG in use is sufficiently good for this purpose. This matter is beyond the scope of 
this document, however default RNG in R, Marsenne-Twister is an appropriate choice.

\subsection{Transition Probability}

The transition probability associated to single spin flip is computed via computing the Boltzmann factor of total energy difference
as follows:

$$p(\{s_{i}\}_{i=1}^{N}) = \exp(-k  \Delta H) \big( 1 + \exp(-k  \Delta H) \big)$$

where $k = \frac{1}{k_{B}T}$ is the Boltzmann factor and $\Delta H$ is the total energy difference.

\subsection{Metropolis Monte Carlo}

In metropolis Monte Carlo, the above transition probability is compared with a uniform number between $[0, 1]$
to check the new lattice configuration induced by the spin flip is an acceptable move. Note that the above formulation
of the transition probability numerically ensures that transition probability lies in between $[0,1]$. This procedure
mimics importance sampling. This is a special case of Metropolis-Hastings Markov Chain Monte Carlo (MCMC).

\section{Utilities}

Package provides utilities to perform Metropolis MC for two state 1D $N$-site lattice, i.e., Ising Model. Functions with suffices with 
$\_R$ are pure R implementations. In this section we document only C based implementations, while functionality is the same. 

One can generate a random configuration, perform single flip on the given configuration, compute nearest-neighbour energy and
total energy. In the following example we generate 7 sites randomly and perform a single spin flip. We compute energies

\begin{Schunk}
\begin{Sinput}
> require(isingLenzMC)
> set.seed(123456)
> N <- 7 
> myInitialConfig <- genConfig1D(N)
> myInitialConfig 
\end{Sinput}
\begin{Soutput}
[1]  1  1 -1 -1 -1 -1  1
\end{Soutput}
\begin{Sinput}
> myNextConfig    <- flipConfig1D(myInitialConfig)
> myNextConfig   
\end{Sinput}
\begin{Soutput}
[1] -1  1 -1 -1 -1 -1  1
\end{Soutput}
\begin{Sinput}
> # nearest neighbour energy for initial config
> lattice1DenergyNN(myInitialConfig) 
\end{Sinput}
\begin{Soutput}
[1] 3
\end{Soutput}
\begin{Sinput}
> # transition probability  at J=H=1/kBT=1.0
> transitionProbability1D(1.0, myInitialConfig, myNextConfig, 1.0, 1.0)
\end{Sinput}
\begin{Soutput}
[1] 0.002472623
\end{Soutput}
\end{Schunk}

It is possible to do the above steps in one go by applying MC move
\begin{Schunk}
\begin{Sinput}
> require(isingLenzMC)
> set.seed(123456)
> N <- 7 
> myInitialConfig <- genConfig1D(N)
> myInitialConfig
\end{Sinput}
\begin{Soutput}
[1]  1  1 -1 -1 -1 -1  1
\end{Soutput}
\begin{Sinput}
> # 1 step Monte Carlo move
> isStep1D(1.0, myInitialConfig, 1.0, 1.0)
\end{Sinput}
\begin{Soutput}
$vec
[1]  1  1 -1 -1 -1 -1  1

$accept
[1] 0
\end{Soutput}
\end{Schunk}

\section{Doing Some Physics}
\subsection{Free Energy}
The partition function $Z_{N}$ can be computed by using the eigenvalues $\lambda_{1}$ and $\lambda_{2}$ of the transfer matrix. 
$$Z_{N}(J, H) = \lambda_{1}^{N} +  \lambda_{2}^{N} $$

For example for 7 sites

\begin{Schunk}
\begin{Sinput}
> Tm <- transferMatrix(1.0, 1.0, 1.0)
> # Free Energy
> log(Tm$evalues[1]^7 + Tm$evalues[2]^7)
\end{Sinput}
\begin{Soutput}
[1] 14.01997
\end{Soutput}
\end{Schunk}
\subsection{Magnetisation: Finite Size Effects}

The average magnetisation of the 1D ising model is simply defined as the average of the lattice site values in the finite case.
However, in theory magnetisation of 1D bulk system ($N \to \infty $) is analytically known
$$ M_{ensemble}(H, T) = exp(J/k_{B} T) sinh(H/k_{B} T) \Big( exp(2K) sinh^{2}(J/k_{B}T) + exp(-2J/k_{B}T) \Big)^{-1/2}$$

This is approximately $0.9934346$ for $J=H=1/k_{B}T=1.0$. This value represents the ensemble average magnetisation for the bulk system.

Now if we simulate a long enough lattice, we see that simulated magnetisation, time average value,  approaches the ensemble average magnetisation value.
\begin{Schunk}
\begin{Sinput}
> require(isingLenzMC)
> set.seed(123456)
> ensembleM <- 0.9934346
> N         <- 200
> x         <- genConfig1D(N)
> mcData    <- isPerform1D(1.0, x, 1.0, 1.0, 10000, ensembleM)
\end{Sinput}
\end{Schunk}
a member of {\it mcData} named list {\it omegaM} reports so called a fluctuation metric over accepted steps, and it is defined as 
follows
$$ \Omega_{M}(k) = \Big( (\sum_{i=1}^{k} M_{ave}^{i}) - M_{ensemble} \Big)^{2} $$
where $M_{ave}^{i}$ is the average magnetisation per site at a given accepted step $i$. It is left as an exercise to see how this 
fluctuation metric changes with the system size $N$. The larger the lattice, longer simulation needed to reach to ensemble average.




\end{document}
